\documentclass[a4paper]{article}
\usepackage[margin=1in]{geometry} % full-width
\usepackage[utf8]{inputenc}
\usepackage{hyperref}
\usepackage{graphicx, color}

\title{Online Bookstore JSON API}
\author{Bradley Marques}

\begin{document}
\maketitle

\newpage

\begin{abstract}
	TODO
\end{abstract}

\newpage

\tableofcontents

\newpage

\section{Introduction}

TODO

\section{Problem Analysis}

\subsection{Domain}

The domain contains the following models:

\begin{itemize}
  \item User - (a.k.a. Author) the user of the system. Has API credentials.
  \item API Credentials - decomposed from user (see justification later).
  \item Book - a Book that an Author has written and wishes to self-publish.
\end{itemize}

\subsection{Models and Database design}

Some notes:

\begin{itemize}
  \item To prevent database walking and therefore increase application security,
  universally unique identifiers (uuids) are used instead of sequential database
  ids
  \item API Credentials are decoupled from Users. Benefits of doing so include ...
\end{itemize}

\subsection{API Endpoints}

\subsection{Assumptions Made}

The following assumptions are made for the purposes of the proof-of-concept:

\begin{itemize}
  \item There is a mandatory 1:1 relationship between a User and an API Credential.
  \item There is a 1:many relationship between a User and Books (i.e. Books cannot have multiple Authors).
  \item Book prices are in United States of America Dollar (USD).
\end{itemize}

\section{Choice of Technology}

The list below summarizes the choice of technology used and gives a brief
justification for each choice:

\begin{itemize}
  \item \textbf{Containerization}: Docker.
  \item \textbf{Programming Language}: Ruby 3.1.0 - the latest stable version at time of writing.
  \item \textbf{Framework}: \href{https://rubyonrails.org/}{Ruby on Rails 7.0.1}: chosen for familiarity and therefore speed of development. Latest stable version at time of writing.
  \item \textbf{API standard/specification}: \href{https://jsonapi.org/}{JSON API}: gold-standard for JSON API responses.
  \item \textbf{Database}: \href{https://www.postgresql.org/}{PostgreSQL}: powerful and feature-rich open-source relational database
\end{itemize}

Rails specific gems used:

\begin{itemize}
  \item \textbf{Authentication}: JWT as specified by brief as well as \href{https://github.com/heartcombo/devise}{Devise}.
  \item \textbf{Authorization}: \href{https://github.com/varvet/pundit}{Pundit} - gem for simple OOP authorization.
  \item ...etc
\end{itemize}

\section{Technical Trade-offs and Future Work}

\begin{itemize}
  \item Monolithic Application
  \item Multi-author books
  \item Bookstore as a model
  \item \textbf{Introduce User roles} - at the moment, a User is by default an Author. However, as
  the system grew, it would be prudent to add the idea of a ``role'' such that
  users could be admins to manage the system, bookstore owners to manage stock,
  etc.
\end{itemize}

\section{Conclusion}

TODO

\end{document}
